\documentclass[a4paper, 12pt]{article}

\usepackage{cmap}
\usepackage[T2A]{fontenc}
\usepackage[utf8]{inputenc}
\usepackage[english, russian]{babel}

\usepackage[left=3cm,right=1cm,top=2cm,bottom=2cm]{geometry}
\setlength{\parindent}{1.25cm}
\usepackage{setspace}
\onehalfspacing

\usepackage{xcolor}
\usepackage[colorlinks, linkcolor=gray]{hyperref}
\usepackage{graphicx}
\usepackage[labelsep=period,justification=centering]{caption}

\title{ИТ. 3 семестр}
\author{Do not die}
\date{\today}

\begin{document}

\maketitle
\tableofcontents
\newpage

% ---------------------------------------------------------------

\section[Информационные технологии. Основные понятия (!!!)]{Информационные технологии. Основные понятия (!!!)\protect\footnote{Часто обсуждаемые вопросы с Барчевым, т.е. он может их задать, даже если этого вопроса нет в билете}}
\label{sec:it-definitions}
\subsection*{Информационные технологии}
\fbox{\large{Опр}} Информационные технологии --- это методы и процессы:
\begin{enumerate}
    \item поиска (поисковые системы),
    \item сбора (получения; датчики температур),
    \item хранения (база данных, знаний),
    \item преобразования (обработки; алгоритм сортировки),
    \item передачи (СМИ)
\end{enumerate}
информации и способы осуществления таких методов и процессов.

\subsection*{Основные понятия}
\subsubsection*{Информация}
\fbox{\large{Опр}} Информация --- это сведения (сообщения, данные), не зависящие от формы их представления:
\begin{enumerate}
    \item[\textbigcircle] Сообщение --- это конкретная форма передачи (предоставления) информации от источника к получателю (текст в чате, звук сирены, сигнал светофора).
    \item[\textbigcircle] Данные -- это сообщение, приведенное к формализованному виду, пригодному для хранения и обработки (JSON-файл: формат представления данных в виде пар ключ-значение).
\end{enumerate}

\subsubsection*{Доступ к информации}
\fbox{\large{Опр}} Доступ --- это возможность получения и использования информации:
\begin{enumerate}
    \item легальный (доступ к собственному аккаунту),
    \item нелегальный (доступ к чужому акканту в результате взлома).
\end{enumerate}

\subsubsection*{Передача информации: распространение и предоставление}
\fbox{\large{Опр}} Предоставление --- передача информации определенному кругу лиц (письмо или имейл лицу). \\
\fbox{\large{Опр}} Распространение --- передача информации неопределенному кругу лиц (пост в соцести, афиша).

\newpage

% ---------------------------------------------------------------

\section{Информационные технологии. Основные направления развития}

\subsection*{Информационные технологии}
\hyperref[sec:it-definitions]{См. вопрос 1.}

\subsection*{Основные направления}
\begin{enumerate}
    \item[\textbigcircle] \hyperref[sec:dev]{Разработка ПО},
    \item[\textbigcircle] \hyperref[sec: maintenance]{Сопровождение ПО},
    \item[\textbigcircle] \hyperref[sec: net]{Сетевые технологии},
    \item[\textbigcircle] \hyperref[sec: internet]{Интернет-технологии},
    \item[\textbigcircle] ... 
\end{enumerate}

\newpage

% ---------------------------------------------------------------

\section{Разработка ПО как одно из направлений информационных технологий (!!!)}
\label{sec:dev}

\subsection*{Разработка ПО}
\fbox{\large{Опр}} Разработка ПО --- систематический процесс создания программного продукта --- совокупности компьютерных программ для решения конкретных задач.

\subsection*{Виды ПО}
\subsubsection*{Классификация по назначению:}
\begin{enumerate}
    \item[\textbigcircle] СПО (системное) --- ПО, обеспечивающее функционирование вычислительной машины (ОС, драйверы, утилиты).
    \item[\textbigcircle] ППО (прикладное), или приложение --- ПО, предназначенное для выполнения конкретных задач пользователя (текстовые редакторы, браузеры, игры).
\end{enumerate}

\subsubsection*{Классификация по способу распространения:}
\begin{enumerate}
    \item[\textbigcircle] Коробочное ПО --- программный продукт, предназначенный для широкого круга пользователей (Paint, Microsoft Office).
    \item[\textbigcircle] Заказное ПО --- ПО, разработанное специально под требования заказчика.
\end{enumerate}

\subsection*{Проектирование ПО}
\fbox{\large{Опр}} Проектирование ПО --- это процесс разработки структуры, архитектуры и принципов взаимодействия компонентов программной системы до начала ее реализации.

\subsection*{Этапы проектирования ПО}
\begin{enumerate}
    \item Проектирование архитектуры.
    \item Программная реализация архитектурных решений.
    \item Вопрос обеспечения качества разработки (тестирование, отладка, анализ).
    \item \hyperref[sec: maintenance]{Сопровождение}.
    \item Организация и автоматизация процесса разработки.
\end{enumerate}

\subsection*{Дополнительно\protect\footnote{Под этим заголовком содержатся термины, про которых нет отдельного вопроса в билете, однако их могут вполне спросить и без того, поэтому стоит аккуратно подбирать слова, чтобы у Барчева не было возможности распрашивать тебя про каждое слово. Термины беру не по прогнозу, а из опыта прошедшего зачета, т.е. это действительно спрашивали}}
\fbox{\large{Опр}} ОС (операционная система) --- СПО, управляющее ресурсами компьютера (оперативной памятью, файловой системой) и обеспечивающее взаимодействие пользователя и прикладных программ с аппаратной частью.\\
\fbox{\large{Опр}} Драйвер --- ПО, обеспечивающее взаимодействие ОС с конкретным аппаратным устройствами (клавиатура, мышь, видеокарта, принтер). \\
\fbox{\large{Опр}} Утилита --- вспомогательная программа для выполнения типовых служебных задач.\\
\fbox{\large{Опр}} ЭВМ (электронно-вычислительная машина) --- совокупность аппаратно-программных средств, предназначенных для автоматической обработки информации.\\
\fbox{\large{Опр}} Архитектура ПО -- структура программной системы и принципы взаимодействия ее компонентов (монолит, клиент-сервер).

\newpage

% ---------------------------------------------------------------

\section{Сопровождение ПО как одно из направлений информационных технологий}
\label{sec: maintenance}

\subsection*{Сопровождение}
\fbox{\large{Опр}} Сопровождение --- совокупность работ, выполняемых после ввода программной системы (ПО и аппаратное обеспечение) в эксплуатацию с целью обеспечения ее стабильной работы и развития.

\subsection*{Виды сопровождения}
\begin{enumerate}
    \item[\textbigcircle] Корректирующее (изменение выявленных ошибок, привлекших за собой дефект, из-за которого у пользователя случился баг).
    \item[\textbigcircle] Адаптивное (модификация системы в связи с изменением внешней среды: законодательство, обновленное оборудование).
    \item[\textbigcircle] Совершенствующее, или эволюционное (добавление новых функций и улучшение существующих -- обновление).
    \item[\textbigcircle] Профилактическое, или предупредительное (работы по предотвращению потенциальных, но нетекущих проблем, по уменьшению технического долга: переписывание легаси-модулей, рефакторинг сложного кода, обновление устаревших или небезопасных библиотек).
\end{enumerate}

\subsection*{Дополнительно}
\fbox{\large{Опр}} Рефакторинг --- изменение программного кода без изменения его поведения в профилактических целях (качество, производительность).\\
\fbox{\large{Опр}} Технический долг --- накопленные архитектурные и кодовые решения, принятые в ущерб качеству ради ускорения разработки, которые впоследствии требуют дополнительных затрат на исправление.\\
\fbox{\large{Опр}} SLA (Service Level Agreement) --- соглашение об уровне обслуживания программной системы, определяющее параметры его качества и срока поддержки.\\
\fbox{\large{Опр}} Легаси --- эксплуатируемый программый код, созданный на устаревших технологиях или с архитектурными ограничениями, затрудняющими ее сопровождение.

\newpage

% ---------------------------------------------------------------
\section{Технологии сетевого взаимодействия как одно из направлений информационных технологий}
\label{sec: net}

\subsection*{Технологии сетевого взаимодействия}
\fbox{\large{Опр}} Технологии сетевого взаимодействия (сетевые технологии) --- это методы и средства организации обмена данными между компьютерами по сети.

\subsection*{Направления деятельности}
\begin{enumerate}
    \item[\textbigcircle] Организация и администрирование сетей: создание, настройка и поддержка работы комьютерных сетей (настройка локальной сети в организациях).
    \item[\textbigcircle] Разработка приложений распределенных, т.е. таких, что все его компоненты работают независимо друг от друга на разных компьютерах как единое целое, что достигается взаимодействием по сети (клиент, сервер и база данных на 3 разных компьютерах).
\end{enumerate}

\subsection*{Дополнительно}
\fbox{\large{Опр}} Компьютерная сеть --- совокупность взаимосвязанных компьютеров и устройств, обеспечивающих обмен данными между ними (LAN и WLAN -- локальные проводная и беспроводная: Ethernet и Wi-Fi). \\
\fbox{\large{Опр}} Интернет --- глобальная система взаимосвязанных комьютерных сетей, использующих единый набор протоколов передачи данных. \\
\fbox{\large{Опр}} Протокол --- набор правил обмена данными между устройствами в сети.

\newpage

% ---------------------------------------------------------------

\section{Интернет-технологии как одно из направлении  
информационных технологий}
\label{sec: internet}

\end{document}
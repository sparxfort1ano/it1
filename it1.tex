\documentclass[a4paper, 12pt]{article}

\usepackage{cmap}
\usepackage[T2A]{fontenc}
\usepackage[utf8]{inputenc}
\usepackage[english, russian]{babel}

\usepackage[left=3cm,right=1cm,top=2cm,bottom=2cm]{geometry}
\setlength{\parindent}{1.25cm}
\usepackage{setspace}
\onehalfspacing

\usepackage{xcolor}
\usepackage[colorlinks, linkcolor=gray]{hyperref}
\usepackage{graphicx}
\usepackage[labelsep=period,justification=centering]{caption}

\title{ИТ. 3 семестр}
\author{Do not die}
\date{\today}

\begin{document}

\maketitle
\tableofcontents
\newpage

% ---------------------------------------------------------------

\section[Информационные технологии. Основные понятия (!!!)]{Информационные технологии. Основные понятия (!!!)\protect\footnote{Часто обсуждаемые вопросы с Барчевым, т.е. он может их задать, даже если этого вопроса нет в билете}}
\label{sec:it-definitions}
\subsection*{Информационные технологии}
\fbox{\large{Опр}} Информационные технологии --- это методы и процессы:
\begin{enumerate}
    \item поиска (поисковые системы),
    \item сбора (получения; датчики температур),
    \item хранения (база данных, знаний),
    \item преобразования (обработки; алгоритм сортировки),
    \item передачи (СМИ)
\end{enumerate}
информации и способы осуществления таких методов и процессов.

\subsection*{Основные понятия}
\subsubsection*{Информация}
\fbox{\large{Опр}} Информация --- это сведения (сообщения, данные), не зависящие от формы их представления:
\begin{enumerate}
    \item[\textbigcircle] Сообщение --- это конкретная форма передачи (предоставления) информации от источника к получателю (текст в чате, звук сирены, сигнал светофора).
    \item[\textbigcircle] Данные -- это сведение, приведенное к формализованному виду, пригодному для хранения и обработки (JSON-файл: формат представления данных в виде набора пар ключ-значение).
\end{enumerate}

\subsubsection*{Доступ к информации}
\fbox{\large{Опр}} Доступ --- это возможность получения и использования информации:
\begin{enumerate}
    \item легальный (доступ к собственному аккаунту),
    \item нелегальный (доступ к чужому акканту в результате взлома).
\end{enumerate}

\subsubsection*{Передача информации: распространение и предоставление}
\fbox{\large{Опр}} Предоставление --- передача информации определенному кругу лиц (письмо или имейл лицу). \\
\fbox{\large{Опр}} Распространение --- передача информации неопределенному кругу лиц (пост в соцести, афиша).

\newpage

% ---------------------------------------------------------------

\section{Информационные технологии. Основные направления развития}

\subsection*{Информационные технологии}
\hyperref[sec:it-definitions]{См. вопрос 1.}

\subsection*{Основные направления}
\begin{enumerate}
    \item[\textbigcircle] \hyperref[sec:dev]{Разработка ПО},
    \item[\textbigcircle] \hyperref[sec: maintenance]{Сопровождение ПО},
    \item[\textbigcircle] \hyperref[sec: net]{Сетевые технологии},
    \item[\textbigcircle] \hyperref[sec: internet]{Интернет-технологии},
    \item[\textbigcircle] \hyperref[sec: web]{Веб-технологии},
    \item[\textbigcircle] \hyperref[sec: media]{Технологии работы с медиаконтентом},
    \item[\textbigcircle] \hyperref[sec: ai]{AI и ML},
    \item[\textbigcircle] \hyperref[sec: is]{Информационная безопасность},
    \item[\textbigcircle] \hyperref[sec: huge]{Крупные комплексные решения}.
\end{enumerate}

\newpage

% ---------------------------------------------------------------

\section{Разработка ПО как одно из направлений информационных технологий (!!!)}
\label{sec:dev}

\subsection*{Разработка ПО}
\fbox{\large{Опр}} Разработка ПО --- систематический процесс создания программного продукта --- совокупности компьютерных программ для решения конкретных задач.

\subsection*{Виды ПО}
\subsubsection*{Классификация по назначению:}
\begin{enumerate}
    \item[\textbigcircle] СПО (системное) --- ПО, обеспечивающее функционирование вычислительной машины (ОС, драйверы, утилиты).
    \item[\textbigcircle] ППО (прикладное), или приложение --- ПО, предназначенное для выполнения конкретных задач пользователя (текстовые редакторы, браузеры, игры).
\end{enumerate}

\subsubsection*{Классификация по способу распространения:}
\begin{enumerate}
    \item[\textbigcircle] Коробочное ПО --- программный продукт, предназначенный для широкого круга пользователей (Paint, Microsoft Office).
    \item[\textbigcircle] Заказное ПО --- ПО, разработанное специально под требования заказчика.
\end{enumerate}

\subsection*{Проектирование ПО}
\fbox{\large{Опр}} Проектирование ПО --- это процесс разработки архитектуры программной системы до начала ее реализации.

\subsection*{Этапы разработки ПО}
\begin{enumerate}
    \item Проектирование архитектуры.
    \item Программная реализация архитектурных решений.
    \item Вопрос обеспечения качества разработки (тестирование, отладка, анализ).
    \item \hyperref[sec: maintenance]{Сопровождение}.
    \item Организация и автоматизация процесса разработки (распределение ролей и постановка задачи -- как, автотесты, автопроверка кода, автосборка и автодеплой).
\end{enumerate}

\subsection*{Дополнительно\protect\footnote{Под этим заголовком содержатся термины, про которых нет отдельного вопроса в билете, однако их могут вполне спросить и без того, поэтому стоит аккуратно подбирать слова, чтобы у Барчева не было возможности распрашивать тебя про каждое слово. Термины беру не по прогнозу, а из опыта прошедшего зачета, т.е. это действительно спрашивали}}
\fbox{\large{Опр}} ОС (операционная система) --- СПО, управляющее ресурсами компьютера (оперативной памятью, файловой системой) и обеспечивающее взаимодействие пользователя и прикладных программ с аппаратной частью.\\
\fbox{\large{Опр}} Драйвер --- ПО, обеспечивающее взаимодействие ОС с конкретным аппаратным устройствами (клавиатура, мышь, видеокарта, принтер). \\
\fbox{\large{Опр}} Утилита --- вспомогательная программа для выполнения типовых служебных задач.\\
\fbox{\large{Опр}} ЭВМ (электронно-вычислительная машина) --- совокупность аппаратно-программных средств, предназначенных для автоматической обработки информации.\\
\fbox{\large{Опр}} Архитектура ПО --- структура программной системы и принципы взаимодействия ее компонентов (монолит, клиент-сервер).

\newpage

% ---------------------------------------------------------------

\section{Сопровождение ПО как одно из направлений информационных технологий}
\label{sec: maintenance}

\subsection*{Сопровождение}
\fbox{\large{Опр}} Сопровождение --- совокупность работ, выполняемых после ввода программной системы (ПО и аппаратное обеспечение) в эксплуатацию с целью обеспечения ее стабильной работы и развития.

\subsection*{Виды сопровождения}
\begin{enumerate}
    \item[\textbigcircle] Корректирующее (исправление выявленных ошибок, привлекших за собой дефект, из-за которого у пользователя случился баг).
    \item[\textbigcircle] Адаптивное (модификация системы в связи с изменением внешней среды: законодательство, обновленное оборудование).
    \item[\textbigcircle] Совершенствующее, или эволюционное (добавление новых функций и улучшение существующих -- обновление).
    \item[\textbigcircle] Профилактическое, или предупредительное (работы по предотвращению потенциальных, но нетекущих проблем, по уменьшению технического долга: переписывание легаси-модулей, рефакторинг сложного кода, обновление устаревших или небезопасных библиотек).
\end{enumerate}

\subsection*{Дополнительно}
\fbox{\large{Опр}} Рефакторинг --- изменение программного кода без изменения его поведения в профилактических целях (качество, производительность).\\
\fbox{\large{Опр}} Технический долг --- накопленные архитектурные и кодовые решения, принятые в ущерб качеству ради ускорения разработки, которые впоследствии требуют дополнительных затрат на исправление.\\
\fbox{\large{Опр}} SLA (Service Level Agreement) --- соглашение об уровне обслуживания программной системы, определяющее параметры его качества и срока поддержки.\\
\fbox{\large{Опр}} Легаси --- эксплуатируемый программый код, созданный на устаревших технологиях или с архитектурными ограничениями, затрудняющими ее сопровождение.

\newpage

% ---------------------------------------------------------------
\section{Технологии сетевого взаимодействия как одно из направлений информационных технологий}
\label{sec: net}

\subsection*{Технологии сетевого взаимодействия}
\fbox{\large{Опр}} Технологии сетевого взаимодействия (сетевые технологии) --- это методы и средства организации обмена данными между устройствами по сети.

\subsection*{Направления деятельности}
\begin{enumerate}
    \item[\textbigcircle] Организация и администрирование сетей: создание, настройка и поддержка работы комьютерных сетей (настройка локальной сети в организациях).
    \item[\textbigcircle] Разработка приложений распределенных, т.е. таких, что все его компоненты работают независимо друг от друга на разных компьютерах как единое целое, что достигается взаимодействием по сети (клиент, сервер и база данных на 3 разных компьютерах).
\end{enumerate}

\subsection*{Дополнительно}
\fbox{\large{Опр}} Компьютерная сеть --- совокупность взаимосвязанных компьютеров и устройств, обеспечивающих обмен данными между ними (LAN и WLAN -- локальные проводная и беспроводная: Ethernet и Wi-Fi). \\
\fbox{\large{Опр}} Интернет --- глобальная система взаимосвязанных комьютерных сетей, использующих единый набор протоколов передачи данных. \\
\fbox{\large{Опр}} Протокол --- набор правил обмена данными между устройствами в сети.

\newpage

% ---------------------------------------------------------------

\section{Интернет-технологии как одно из направлении  
информационных технологий}
\label{sec: internet}

\subsection*{Интернет-технологии}
\fbox{\large{Опр}} Интернет-технологии --- это методы и средства организации передачи данных и функционирования информационных систем в сети Интернет.

\subsection*{Направления деятельности}
\begin{enumerate}
    \item[\textbigcircle] \hyperref[sec: web]{Веб-разработка}.
    \item[\textbigcircle] Поддержка информационных ресурсов: администрирование сайтов и серверов, обновление контента, обеспечение безопасности и доступности ресурсов.
    \item[\textbigcircle] Облачные технологии --- предоставление вычислительных ресурсов (на разворачивание -- установку, настройку и запуск -- сервера, поднятие -- создание и запуск -- базы данных) через сеть Интернет.
    \item[\textbigcircle] \hyperref[sec: media]{Медиаконтент}.
    \item[\textbigcircle] Поисковые технологии и оптимизация поиска ресурсов --- разработка алгоритмов поиска, индексации и ранжирования (сортировки по определенным критериям) информации.
\end{enumerate}

\subsection*{Дополнительно}
\fbox{\large{Опр}} Интернет --- глобальная система взаимосвязанных комьютерных сетей, использующих единый набор протоколов передачи данных. \\
\fbox{\large{Опр}} Индексирование информации --- процесс сканирования, анализа и добавления данных со страниц сайта в базу данных поисковых систем (Гугл, Яндекс). \\
\fbox{\large{Опр}} DNS (Domain Name System) --- система доменных имен, обеспечивающая соответствие между доменным имененем и IP-адресом сервера. \\
\fbox{\large{Опр}} Домен (Доменное имя) --- это символьное имя, используемое для идентификации ресурса в сети Интернет. \\
\fbox{\large{Опр}} IP-адрес --- это уникальный числовой идентификатор устройства в сети, использующий протокол IP, предназначенный для адресации и доставки данных.

\newpage

% ---------------------------------------------------------------

\section{ Веб-технологии как направления информационных 
технологий. Веб-разработка (!!!)}
\label{sec: web}

\subsection*{Веб-технологии}
\fbox{\large{Опр}} Веб-технологии --- это технологии создания и фукционирования гипертекстовых информационных ресурсов (предназначенные для хранения и предоставления информации, статический: новостной сайт, электронная библиотетка) и веб-приложений (обеспечивающие интерактивное взаимодействие пользователя с сервером, динамический: почтовый сервис, онлайн-магазин) в среде WWW.

\subsection*{Веб-разработка} 
\fbox{\large{Опр}} Веб-разработка --- это процесс создания веб-сайтов и веб-приложений. Разделяется на:
\begin{enumerate}
    \item Клиентскую разработку (Front-end), отвечающую за внешнюю, клиентскую часть веб-сайтов и приложений: отображение информации и взимодействие с ней. Используются HTML, CSS, JavaScript.
    \item Серверную разработку (Back-end), отвечающую за обработку запросов клиента в рамках бизнес-логики, безопасности, хранения данных в базе данных и т.д. Используются PHP, Python, Java, Go.
\end{enumerate}

\subsection*{Дополнительно}
\fbox{\large{Опр}} WWW (World Wide Web) --- это распределенная система гипертекстовых документов, ресурсов, доступных через сеть Интернет с использованием HTTP. \\
\fbox{\large{Опр}} HTTP (HyperText Transfer Protocol) --- это протокол, использующийся для передачи данных в сети Интернет. \\
\fbox{\large{Опр}} Гипертекст --- способ организации информации в виде взаимосвязанных документов или их фрагментов, между которыми возможен переход с помощью гиперссылок. \\
\fbox{\large{Опр}} Гиперссылка --- элементов гипертекста, обеспечивающих переход к другому документу или его части. \\
\fbox{\large{Опр}} HTML (HyperText Markup Language) --- язык разметки гипертекста, предназначенный для структурирования информации на веб-странице. \\
\fbox{\large{Опр}} Клиент --- программа, инициирующая запрос к серверу ддя получения данных или услуг (в частности, браузер). \\
\fbox{\large{Опр}} Сервер --- программа, принимающая и обрабатывающая запросы от клиентов для предоставления им ответных данных. 

\newpage

% ---------------------------------------------------------------

\section{Технологии работы с медиаконтентом как направление информационных технологий}
\label{sec: media}

\subsection*{Технологии работы с медиаконтентом. Медиаконтент}
\fbox{\large{Опр}} Технологии работы с медиаконтентом --- технологии создания, обработки, хранения и передачи медиаконтента, анализа мультимедийной информации (объединяющей в себе разные виды медиаконтента). \\
\fbox{\large{Опр}} Медиаконтент --- разнообразная цифровая информация, представленная в аудио- и визуальных форматах (видео, фото).

\subsection*{Направления деятельности}
\begin{enumerate}
    \item[\textbigcircle] Обработка изображений и видео (сжатие (компрессия), улучшение качества).
    \item[\textbigcircle] Поддержка информационных ресурсов: администрирование сайтов и серверов, обновление контента, обеспечение безопасности и доступности ресурсов.
    \item[\textbigcircle] Компьютерное зрение (CV): анализ визуального медиаконтента (в частности, распознавание).
    \item[\textbigcircle] Обработка и синтез речи (распознавание речи, голосовые ассистенты, преобразование текста в речь).
\end{enumerate}

\newpage

% ---------------------------------------------------------------

\section{Базы данных, базы знаний как средства поддержки 
информационных технологий}
\label{sec: db}

\subsection*{База данных}
\fbox{\large{Опр}} База данных (БД) --- это организованная совокупность структурированных данных, предназначенных для хранения, поиска и обработки информации.

\subsubsection*{Управление БД}
Взаимодействие с БД осуществляется через \fbox{\large{СУБД}} (систему управления базой данных): создание таблиц хранения данных, управление доступом, защита данных, выполнение запросов через \fbox{\large{SQL}} (Structured Query Language -- язык структурированных запросов) для управления, создания, модификации и получения данных в БД.

\subsection*{База знаний}
\fbox{\large{Опр}} База знаний --- это структурированная совокупность знаний о предметной области, включающая факты и правила вывода, т.е. хранилище знаний (правила и зависимости, за счет которых заключается по имеющимся данным новые, заключащие вердикт данные).\\\\
Базы знаний используются внутри \fbox{\large{экспертных систем}} (ЭС) --- ППО, использующего базу знаний и механизм логического вывода для решения задач на уровне эксперта.

\subsection*{Прояснение различия}
\begin{enumerate}
    \item[\textbigcircle] Базы данных отвечают на вопрос: ``Какие данные есть?". Например, (Температура двигателя) = 105, (Обороты) = 900.
    \item[\textbigcircle] Базы знаний хранят правила и зависимости, отвечают на вопрос: ``Какие связи существуют между данными?". Например, IF (температура двигателя) > 100, THEN <перегрев>.
    \item[\textbigcircle] ЭС выдает конкретный логический вердикт по выводу базы знаний, отвечает на вопрос: ``Какой вывод можно сделать из меющихся данных?". Например, ``Обнаружен перегрев двигателя. Рекомендуется проверить систему охлаждения."  
\end{enumerate}

\newpage

% ---------------------------------------------------------------

\section{Искусственный интеллект и машинное обучение как одно из направлений информационных технологий}
\label{sec: ai}

\subsection*{Искусственный интеллект}
\fbox{\large{Опр}} Искусственный интеллект (ИИ) --- технологии создания программных систем, способные имитировать когнитивные функции человека: обучение, анализ (в частности, диагностика), распознавание образов (в частности, CV), принятие решений, обработка естественного языка (NLP: анализ и генерация человеческой речи или текстов).

\subsection*{Машинное обучение}
\fbox{\large{Опр}} Машинное обучение (МО) --- это раздел ИИ, основанный на создании алгоритмов, способных обучаться на данных (data) и улучшать результаты без явного программирования правил.

\subsubsection*{Типы МО}
\begin{enumerate}
    \item[\textbigcircle] Обучение с учителем: есть входные данные X, правильные ответы Y, модель учится отображению X ---> Y. В частности, задача классификации: задано конечное множество объектов, для которых известно, к каким классам они относятся. Цель: автоматически отнести новые объекты к одному из заранее известных классов на основе анализа набора признаков. Например, соотнести спам или не спам. 
    \item[\textbigcircle] Обучение без учителя: есть только данные X, правильных ответов нет. В частности, задача кластеризации: заданы объекты, группирующиеся по сходству без заранее заданных классов. Цель: самостоятельно выявить скрытые группы (кластеры) без использования размеченных ответов. Например, есть данные о возрасте, количеству покупок, а алгоритм сам выделяет группы: ``молодые'', ``постоянные клиенты'' и т.д.
\end{enumerate}

\newpage

% ---------------------------------------------------------------

\section{Информационная безопасность как одно из 
направлений информационных технологий}
\label{sec: is}

\subsection*{Информационная безопасность}
\fbox{\large{Опр}} Информационная безопасность --- технологии защиты информации от угроз, утечек и атак, направленные на обеспечение конфиденциальности, целостности и доступности данных.

\subsection*{Основные аспекты}
\begin{enumerate}
    \item[\textbigcircle] Анализ угроз безопасности. 
    \begin{enumerate}
    \item[\textbigcircle] Анализ возможных векторов атак (определение путей, через которые может быть осуществлена атака на систему, например, через уязвимости в ПО).
    \item[\textbigcircle] Анализ способов противодействия (выявление методов защиты от атак, например, антивирусные системы).
    \item[\textbigcircle] Построение модели угроз (создание схемы потенциальных атак и защитных мер, чтобы заранее понимать, какие угрозы могут возникнуть).
    \end{enumerate}
    \item[\textbigcircle] Тестирование на проникновение (пентест) --- метод реальных атак на систему с целью выявления ее слабых мест (атака системы, анализ уязвимости, предложения по усилению защиты).
    \item[\textbigcircle] Защита компьютерных систем:
    \begin{enumerate}
        \item[\textbigcircle] Криптографические мероприятия: шифрование данных, создание безопасных каналов связи.
        \item[\textbigcircle] Организационно-технические мероприятия: политики безопасности, обучение персонала, контроль доступа.
    \end{enumerate}
\end{enumerate}

\subsection*{Дополнительно}
\fbox{\large{Опр}} Идентификация (``кто Вы?'') --- процесс сообщения системе пользователем своего уникального идентификатора (логин, email, номер телефона). \\
\fbox{\large{Опр}} Аутентификация (``точно ли это Вы?'') --- процесс проверки подлинности заявленной личности пользователя (пароль, пин-код, SMS-код, отпечаток пальца, Face ID). \\
\fbox{\large{Опр}} Авторизация (``что Вам разрешено?'') --- процесс предоставления прав доступа пользователю после успешной аутентификации (права пользователя, модератора, админа).

\newpage

% ---------------------------------------------------------------

\section{Крупные комплексные решения как одно из направлений информационных технологий}
\label{sec: huge}

\subsection*{Крупные комплексные решения}
\fbox{\large{Опр}} Крупные комплексные решения --- это системы, состоящие из множества взаимосвязанных компьютеров, работающих совместно для решения масштабных и сложных задач в различных сферах.

\subsection*{Примеры крупных комплексных решений}
\begin{enumerate}
    \item[\textbigcircle] Кластерные системы --- для повышения производительности (HPC -- высокопроизводительные вычисления из кластеров сотен или тысяч связанных серверов для решения математических задач) и отказоустойчивости (HA -- кластер высокой доступности, гарантирующий минимальное время простоя за счет аппаратной избыточности: нагрузки вышедшего из строя узла переносятся на исправные узлы).
    \item[\textbigcircle] Мультимедийные платформы --- для обработки и передачи мультимедийного конктента (платформы стриминга видео: YouTube, Netflix), образовательные платформы (Stepik, LMS).
    \item[\textbigcircle] Поисковые машины --- для поиска информации по запросам (Google, Yandex) с применением алгоритмов ранжирования и индексации.
    \item[\textbigcircle] Системы бронирования --- для резервирования услуг: билеты, отели (Booking.com), аренда автомобилей и т.д.
    \item[\textbigcircle] Инженерные системы --- для решения инженерных задач: моедлирование, проектирование (CAD-системы), управление сложными технологическими процессами (SCADA).
\end{enumerate}

\end{document}